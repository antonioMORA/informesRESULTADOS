%%% Plantilla creada para Proyecto de Investigación
%%% Instituto Tecnologico Superior de Escarcega
%%% Cristel Maria Escalante Garcia
%%% Ver 1.0, 30 Mayo 2017
\documentclass[a4paper,10pt]{report} 

\usepackage[utf8]{inputenc}
\usepackage[spanish]{babel}
\usepackage{amsmath}
\usepackage{amsfonts}
\usepackage{amssymb} 

\usepackage{graphicx} 
\usepackage{hyperref} 
\usepackage{wrapfig}
\usepackage{enumitem}
\usepackage{fancyhdr}
\usepackage{float}
\usepackage{eurosym}
\usepackage{color}
\usepackage{titling}
\usepackage{lipsum}
\usepackage{tocbibind}


\usepackage[left=3cm,right=3cm,top=3cm,bottom=4cm]{geometry}


\pagestyle{fancy}


%%% Para las cabeceras
\newcommand{\hsp}{\hspace{20pt}}
\newcommand{\HRule}{\rule{\linewidth}{0.5mm}}
\headheight=50pt
%%% 
\newcommand{\vacio}{\textcolor{white}{holacaracola}}

%%% Para que las ecuaciones se numeren
%%% con el número de sección y el de ecuación.
\renewcommand{\theequation}{\thesection.\arabic{equation}}


% Color azul para algunos 
% textos de la portada
\definecolor{azulportada}{cmyk}{0.5,1,0,0.1}

%%%% Azul para textos de headings
\definecolor{azulinterior}{cmyk}{0.5,1,0,0.1}

%%%%%%%%%%%%%%%%%%%%%%%%%%%%%%%%
%%%%%% Datos del proyecto %%%%%%
%%%%%%%%%%%%%%%%%%%%%%%%%%%%%%%%
%%%TÍTULO
%%% Escribirlo en minúsculas, el programa
%%% lo pondrá en mayúsculas en la portada.
\title{Impacto del aprendizaje en algebra elemental, mediante un software matemático.}
%%%% AUTOR
\author{Cristel Maria Escalante Garcia}
%%%%%%%%%%%%%%%%
%%%%%% DIRECTOR DEL TRABAJO
%%%%%%% Cambiar el nombre siguiente
\newcommand{\director}{MC. Manuel Arturo Suarez Amendola }
%%%%%%%%%%%%%%%

%%%%%%%%%%%%%%%%%%%%%
%%%%%%%%%%%%%%%%%%%%
\begin{document}

%%%%%%%%%%%%%%%%%%%%%%%%%%%%%%%
%%%%%%%%%%%%%%%%%%%%%%%%%%%%%%%
\begin{titlepage} %%%%% Aquí no hay que tocar nada.
	%%%% Las siguientes instrucciones generarán automáticamente
	%%%% la portada de tu proyecto.
	%%% Cambio de la estructura de esta página
\newgeometry{left=0.6cm,top=1.3cm,bottom=1.2cm}

\fbox{\parbox[c]{18.5cm}{
\begin{center}
\vspace{1.5cm}
{\fontfamily{phv}\fontsize{20}{6}\selectfont{Instituto Tecnológico Superior de Escárcega}}\\
[1em]
{\fontfamily{phv}\fontsize{15}{5}\selectfont{Ingeniería En Sistemas Computacionales}}\\
[1em]
{\fontfamily{phv}\fontsize{18}{5}\selectfont{INFORME DE INVESTIGACIÓN}}\\
[2.6cm]
% Autor del trabajo de investigación
\textcolor{azulportada}{\fontfamily{phv}\fontsize{16}{5}\selectfont{\theauthor}}\\
[1cm]
% Título del trabajo
\textcolor{azulportada}
{\fontfamily{phv}\fontsize{30}{5}\selectfont{\textsc{\thetitle}}}\\
%{\Huge\textbf{\thetitle}}\\
[1cm]
\includegraphics[width=8cm]{logoitse.png}
\\[2cm]
{\fontfamily{phv}\fontsize{16}{5}\selectfont{Trabajo dirigido por}}\\
[0.5cm]
{\fontfamily{phv}\fontsize{16}{5}\selectfont{\director}}\\
[2cm]
{\fontfamily{phv}\fontsize{16}{5}\selectfont{curso Febrero - Julio 2017}}\\
[4cm]
\end{center}
}}
 
 \restoregeometry
 %%%% Volvemos a la estructura de la página normal

\end{titlepage}

%%%%%%%%%%%%%%%%%%%%%%%%%%%%%%

{%\Large

\newpage

%%%Encabezamiento y pie de página
%%% También se genera automáticamente
%%% Mejor no tocarlo mucho.
\renewcommand{\headrulewidth}{0.5pt}
\fancyhead[R]{
	\textcolor{azulinterior}{\fontfamily{phv}\fontsize{14}{4}\selectfont{\textbf{\thetitle}}}\\
\textcolor{azulportada}{\fontfamily{phv}\fontsize{10}{3}\selectfont{Taller de Investigación II -- curso Febrero - Julio 2017}}\\
{\fontfamily{phv}\fontsize{10}{3}\selectfont{\theauthor}}}
\fancyhead[L]{\vacio}

\renewcommand{\footrulewidth}{0.5pt}
\fancyfoot[L]{\footnotesize Instituto TecnolOgico Superior de Escarcega --- curso Febrero - Julio2017}
\fancyfoot[C]{\vacio}
\fancyfoot[R]{\footnotesize Página \thepage}


%%%%%%%%%%%%%%%%%%%%

\

\vacio

\


\subsection*{Resumen}
En no más de cinco líneas se ha de escribir un resumen del contenido del ensayo: qué se pretendía probar/describir/discutir; cuál era el 'estado de la cuestión’; qué método se ha empleado; cuáles son las conclusiones alcanzadas en sustancia.


\subsection*{Abstract}
\textsl{
The content of the paper must be summed up in no more than five lines: what was the aim of the paper; what was the ‘State of the Art’; which method has been used; what are the main conclusions.}

\ %% Así hago que se abra más espacio entre renglones.

\

\hrule

\

\


\paragraph{Palabras clave:} dos o tres palabras o conceptos (grupos de palabras) que permitan identificar el trabajo realizado y encuadrarlo en una ciencia o saber concreto.

\


\paragraph{Keywords:} Two or three words or ideas that allow to frame the paper into a specific science or field of knowledge.







\newpage

%%% En esta página va el índice,
%%% pero no hay que hacer nada porque 
%%% se generará automáticamente.

\tableofcontents

\newpage



\section{\textbf{INTRODUCIÓN}}


%
\subsection{Tema del Proyecto}

\subsection{\textbf{Pregunta de Investigación}}
¿Qué tanto influirá el software matemático en el aprendizaje del alumno sobre el álgebra elemental, en ecuaciones de segundo grado, para alumnos de primero y segundo año de secundaria?

\subsection{Importancia del Estudio}
Es un hecho notorio que las matemáticas ocupan, en casi todos los países, un lugar central en todos los programas escolares. Desde el nivel primaria, suele existir una enseñanza basica sobre la naturaleza de las matemáticas, aunque cada docente lo puede explicar de forma diferente y cada alumno puede entenderlo conforme lo va practicando. Ahora bien, si nos detenemos en las escuelas secundarias, observamos una extraordinaria variedad en el contenido de los cursos. A pesar de la pretendida universalidad de las matemáticas, es posible encontrar países en los que los programas de matemáticas de la escuela secundaria no tienen casi nada en común, lo que nos lleva a preguntarnos: ¿son realmente las matemáticas tan importantes como se pretende? Cuando se examina esta cuestión reina a menudo bastante confusión acerca del sentido en que se utiliza la palabra "matemáticas". Por ello, quizás debiéramos empezar por tratar de aclarar nuestras ideas al respecto. 

\section{MARCO TEÓRICO}

\subsection{Hipótesis}

\subsection{Planteamiento del Problema}

Se pueden hacer secciones, que después aparecerán automáticamente en el índice.

\subsection{Instrumentos de medición}

\section{RESULTADOS}

Y seguirán apareciendo en el índice automáticamente.

\subsection{Graficas de distribuciones}

\subsection{Indices de correlación de las variables}

\subsection{Prueba de Hipótesis}

\section{CONCLUSIONES}

\subsection{Trabajo a Futuro}

\section{REFERENCIAS BIBLIOGRÁFICAS}

\subsection{La bibliografía en página aparte}

Lo mejor es escribir la bibliografía en una página nueva. 

Al igual que ocurre con las ecuaciones, las referencias de la bibliografía se pueden citar en el texto mediante la etiqueta que le pongamos a cada una. (Consultar el fichero fuente para ver cómo). 

Por ejemplo, si quiero referirme al artículo \cite{ausubel}, o bien al libro \cite{bacaer}, lo puedo hacer escribiendo \verb|\cite{etiqueta}|.  

\

\paragraph{Enlaces en la bibliografía:} Si la referencia que ponemos en la bibliografía se puede encontrar en internet, resulta interesante poner, al final, la dirección para que pueda ser consultada. Por supuesto, los enlaces que se pongan deben dirigir a páginas que cumplan con todos los requisitos legales en cuanto al respeto a los derechos de la propiedad intelectual.


\newpage

\renewcommand\refname{Bibliografía}

\begin{thebibliography}{9}

\bibitem[A]{aligica} P.D. Aligica. \textsl{Julian Simon and the “Limits to Growth” Neo-Malthusianism}, Electronic Journal of Sustainable Development, \textbf{1}, 3, (2009), pp. 73-84. (http://goo.gl/23G1Oo)

\bibitem[Au]{ausubel} J. H. Ausubel y P. S. Meyer. \textsl{Carrying Capacity: A Model with
Logistically Varying Limits, Technological Forecasting and Social Change},
\textbf{61}, 3, (1999), pp. 209-214. http://goo.gl/Lpc4g4

\bibitem[Al]{alvarez} N. Álvarez-Vázquez, P.A. Pérez y J. Rodríguez-Ruiz. \textsl{Métodos y
modelos matemáticos de la demografía}, (trabajo), Departamento de Economía
Aplicada Cuantitativa, UNED, Málaga, 1997. http://goo.gl/n1Z2nM

\bibitem[B]{bacaer} N. Bacaër. \textsl{A Short History of Mathematical Population Dynamics},
Springer-Verlag, Londres, 2011. http://goo.gl/1LhzMB

\bibitem[Ba]{banks} R. B. Banks. \textsl{Growth and Diffusion Phenomena: Mathematical
Frameworks and Applications}, Springer-Verlag, Berlín, 1994.


\end{thebibliography}


\newpage

\section*{Agradecimientos}

Esto es por si queremos agradecer a alguien. Aunque no vamos a hacer que aparezca en el índice. 


\end{document}



